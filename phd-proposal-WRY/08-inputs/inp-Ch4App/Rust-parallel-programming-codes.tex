%%% LANDSCAPE BEGINS =======================================
%\pagebreak
%\clearpage
%\begin{flushleft}
%\begin{landscape}

% =========================================================
\subsection{App4-Rust Parallel Multithreading Codes}
% =========================================================
\begin{enumerate}
	\item We provided the Rust program source code: rust-multi-threaded.rs
	
	\item The file suffix (.rs) is for Rust program code. We compile this Rust code using the Rust compiler (rustc)
	
	\item Rust Compiler:\\
	
	wruslan\@HP-ELBook8470p-ub1604-64b: \$ which rustc\\
		/home/wruslan/.cargo/bin/rustc\\
	
	wruslan\@HP-ELBook8470p-ub1604-64b: \$ rustc -version\\
		rustc 1.28.0 (9634041f0 2018-07-30)\\
	
	\item The resulting compiled Rust code does not have a suffix, just rust-multi-threaded.
	
	\item Child threads: In this multi-threaded Rust program, the main thread spawned three(3) child threads: thread01(x), thread02(y) and thread03(z), where x, y and z, are random integers between 1 and 20 (seconds), to simulate the sleep times of each individual child thread.
	
	\item Main thread: The main thread is made to run beyond the longest time for any one of the three(3) child threads to finish. Since each child thread will have a maximum assigned finished time of 20 seconds (sleep), we made the main thread sleep for 21 seconds. This is to ensure that all child threads will finish early and join the main thread.
	
	\item If we reduce the main thread sleep to less than 20 seconds, then some child threads would not have finished by this time. This means when the main thread dies, all the child threads get pulled and die together (whether finished or not yet finished). This is multi-threading.
	
	\item Note that the total execution time is the time the main thread runs active, which is set at 21 seconds. You can reduce this setting, to prove that it is multi-threading (as explained above).
	
	\item Proof : This total time is not the sum of the sleep times of the child threads. If it is the sum, then we have sequential-in-time execution not parallel-in-time execution. This is the proof that all child threads run independently and parallel-in-time.
	
	\item Note that the codes for Rust is much more compact than C/C++ codes. Rust was designed from the ground up with Safety, Security and Concurrency (SSC) built-into the Rust engine, not many separate software libraries, unlike C/C++ language. 
	
\end{enumerate}

\pagebreak

% ==========================================================
\lstset{basicstyle=\footnotesize, numberstyle=\tiny\color{blue}, frame=single, numbers=left, firstnumber=1, stepnumber=1, numbersep=1pt, xleftmargin=2.0em, framexleftmargin=1.5em, xrightmargin=0.0em, breaklines=true, breakatwhitespace=false, breakindent=5pt, prebreak=\space, postbreak=\space }
% ==========================================================

\begin{lstlisting}[caption={App4-Rust Parallel Multithreading Codes}, label=App4-Rust Parallel Multithreading Codes]
// File: rust-multi-threaded.rs
// Date: Sat Aug 25 15:42:16 +08 2018
// Author: WRY
// =======================================================

	use std::thread;
	use std::time::{Duration, SystemTime, UNIX_EPOCH};

// ========================================================
// SEE SOLUTION (1) BY WRY HOW TO SOLVE THIS PROBLEM
// https://crates.io/crates/rand
// About generating random numbers in Rust

	extern crate rand;
	use rand::prelude::*;
	
// ========================================================
// SEE SOLUTION (2) BY WRY HOW TO SOLVE THIS PROBLEM
// https://crates.io/crates/chrono
// About getting current date and time in Rust

	extern crate chrono;
	use chrono::prelude::*;

// ======================
// HIGH RESOLUTION TIME
// ======================

fn hires_time() {
	let start = SystemTime::now();
	let since_the_epoch = start.duration_since(UNIX_EPOCH).expect("Time went backwards");
	print!("{:?} ", since_the_epoch);
}

// SIMPLE THREAD 1
// ===========================
fn my_thread01(x: u64) {
// ===========================
	hires_time(); print!("\tStarting child thread01({:?}).\n", x);
	thread::sleep(Duration::from_millis(1000*x)); 
	hires_time(); print!("\tFinished child thread01({:?}).\n", x);	
}

// SIMPLE THREAD 2
// ===========================
fn my_thread02(y: u64) {
// ===========================
	hires_time(); print!("\tStarting child thread02({:?}).\n", y);
	thread::sleep(Duration::from_millis(1000*y)); 
	hires_time(); print!("\tFinished child thread02({:?}).\n", y);	
}

// SIMPLE THREAD 3
// ===========================
fn my_thread03(z: u64) {
// ===========================
	hires_time(); print!("\tStarting child thread03({:?}).\n", z);
	thread::sleep(Duration::from_millis(1000*z)); 
	hires_time(); print!("\tFinished child thread03({:?}).\n", z);	
}

// ========================================================
fn main() {
// ========================================================

//  TESTED GOOD RANDOM NUMBERS BASED ON SOLUTION(1) ABOVE
// 
	//  let x: u8 = random();
	//  println!("random u8 = {}", x);
	
	//  let y = random::<f64>();
	//  println!("random::<f64> = {}", y);
	
	//  let sleep_time: u8 = thread_rng().gen_range(1, 21);
	//  println!("sleep_time u8 value = {:?}", sleep_time); 

// TESTED GOOD DATE RUN TODAY NOW BASED ON SOLUTION(2) ABOVE
	// let utc: DateTime<Utc> = Utc::now();       
	// println!("DateTime<Utc> value = {}", utc);
	
	// let local: DateTime<Local> = Local::now(); 
	// println!("DateTime<Local> value = {:?}", local);

// ========================================================
// IMPLEMENTATION

	let local: DateTime<Local> = Local::now();
	hires_time(); print!("\tBismillah from WRY. DateTime now = {:?} \n", local);
	
	let sleep_time01: u64 = thread_rng().gen_range(1, 21);
	
	println!("Assigned sleep_time01 u64 value = {:?}", sleep_time01);
	thread::spawn(move || {
		hires_time();
		print!("\tMain thread spawning child thread01({:?})\n", sleep_time01);
		my_thread01(sleep_time01);
	});
	
	let sleep_time02: u64 = thread_rng().gen_range(1, 21);
	println!("Assigned sleep_time02 u64 value = {:?}", sleep_time02);

	thread::spawn(move || {
		hires_time(); 
		print!("\tMain thread spawning child thread02({:?})\n", sleep_time02);
		my_thread02(sleep_time02);
	});

	let sleep_time03: u64 = thread_rng().gen_range(1, 21);
	println!("Assigned sleep_time03 u64 value = {:?}", sleep_time03);

	thread::spawn(move || {
		hires_time(); 
		print!("\tMain thread spawning child thread03({:?})\n", sleep_time03);
		my_thread03(sleep_time03);
	});

// IMPT: Main to sleep longer than all child threads, otherwise the remaining
// child threads die together when main thread dies.
// Try reducing this sleep time and see. Ha ha ha.

	hires_time();
	print!("\tMain thread starting to sleep for max XXX seconds.\n");

	thread::sleep(Duration::from_millis(21000)); // Wait for 21 seconds.
	hires_time();print!("\tMain thread finished sleep time max for 21 seconds.\n");

	hires_time(); print!("\tAlhamdulillah from WRY.\n");
	
} // END main()

// ========================================================
/* COMPILATION 

wruslan@HP-ELBook8470p-ub1604-64b:~/Downloads/temp3$ rustc rust-multi-threaded.rs 

EXECUTION NO. 1
====================
wruslan@HP-ELBook8470p-ub1604-64b:~/Downloads/temp3$ ./rust-multi-threaded 
1535187474.12119863s 	Bismillah from WRY. DateTime now = 2018-08-25T16:57:54.121130393+08:00 
1535187474.123016112s 	Main thread spawning child thread01(4)
1535187474.12304328s 	Starting child thread01(4).
1535187474.123467714s 	Main thread starting to sleep for max 21 seconds.
1535187474.123703762s 	Main thread spawning child thread02(2)
1535187474.123724902s 	Starting child thread02(2).
1535187474.123972992s 	Main thread spawning child thread03(18)
1535187474.123986326s 	Starting child thread03(18).
1535187476.123839552s 	Finished child thread02(2).
1535187478.123198653s 	Finished child thread01(4).
1535187492.124127672s 	Finished child thread03(18).
1535187495.123586666s 	Main thread finished sleep time max for 21 seconds.
1535187495.123627182s 	Alhamdulillah from WRY.
wruslan@HP-ELBook8470p-ub1604-64b:~/Downloads/temp3$ 

ANALYSIS
1535187492.124127672s 	Finished child thread03(18).
-1535187474.123986326s 	Starting child thread03(18).
=======================
18.001  seconds HA HA HA Time for longest thread to finish.

EXECUTION No. 2
======================
wruslan@HP-ELBook8470p-ub1604-64b:~/Downloads/temp3$ ./rust-multi-threaded 
1535187502.325989233s 	Bismillah from WRY. DateTime now = 2018-08-25T16:58:22.325920986+08:00 
1535187502.32760476s 	Main thread starting to sleep for max 21 seconds.
1535187502.32786682s 	Main thread spawning child thread01(20)
1535187502.327892862s 	Starting child thread01(20).
1535187502.328241946s 	Main thread spawning child thread02(10)
1535187502.328266967s 	Starting child thread02(10).
1535187502.328634609s 	Main thread spawning child thread03(17)
1535187502.328649296s 	Starting child thread03(17).
1535187512.328427258s 	Finished child thread02(10).
1535187519.328750721s 	Finished child thread03(17).
1535187522.328023394s 	Finished child thread01(20).
1535187523.327788422s 	Main thread finished sleep time max for 21 seconds.
1535187523.32783525s 	Alhamdulillah from WRY.
wruslan@HP-ELBook8470p-ub1604-64b:~/Downloads/temp3$ 

ANALYSIS
1535187522.328023394s 	Finished child thread01(20).
-1535187502.327892862s 	Starting child thread01(20).
=======================
20.001  seconds HA HA HA Time for longest thread to finish. 

ALHAMDULILLAH 3 TIMES.
*/
// ========================================================
\end{lstlisting}

% =========================================================
%\clearpage
%\end{landscape}
%\end{flushleft}
